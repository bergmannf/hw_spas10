%************************************************
\chapter{Conclusions}\label{ch:conclusions} % $\mathbb{ZNR}$
%************************************************

To conclude this report a short summary of the achieved goals (apart from fulfilling the requirements) should be provided:

The application's design tries to be robust, yet adaptable if changes need to be made.
To ensure the achieving of these goals, patterns were used where it seemed appropriate.

Moreover the loose coupling should help in the achieving of these goals as well.

The approach to provide a well-designed application from a software engineering point-of-view may  have prevented the addition of many convenience features. However, it seemed more important to deliver an easily adaptable program: exchanging the \ac{GUI} or the persistence mechanism should be fairly easy.
The presenters are already in place thus that only the \ac{GUI} classes need to be modified
If the persistence mechanism should be changed the only thing that needs to be provided is a new implementation that adheres to the \texttt{ISerialiser<T>} interface.

However concerning the persistence, there is one major flaw in the application: the fact that the \texttt{ApplicationSettings} need to be persisted in the WinForms-project. 
The development of a small framework to allow a similar mechanism in other projects might prevent this circumstance in future projects.

Overall the application should fulfil the requirements, which seems to be the most important goal to achieve in the first place.