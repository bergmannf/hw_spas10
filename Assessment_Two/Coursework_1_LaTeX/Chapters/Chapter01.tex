%************************************************
\chapter{Introduction}\label{ch:introduction}
%************************************************

This chapter will provide an overview over the document and recapture the requested requirements.

\section{Document overview}
\label{sec:document_overview}

The document is divided into seven chapters that will describe different aspects of the developed program:

\autoref{ch:introduction} provides an overview over the document and the specified requirements, alongside certain assumptions that were made during the development.

\autoref{ch:requirements} will give a short outline of the requirements that were fulfilled, as well as mention added functionality that was not requested.

\autoref{ch:design} will provide a high level overview over the system's architecture and its sub-modules.

\autoref{ch:user_guide} is a short user guide that will describe the usage of the program by leading the reader through a selected choice of use cases to accomplish certain tasks.

\autoref{ch:developer_guide} will be based upon \autoref{ch:design} and provide an in-depth explanation implementation-details.

\autoref{ch:testing} will outline how testing was performed and which cases have been covered.

\autoref{ch:conclusions} will provide a reflection of the development process and the program and highlight areas of interest from the developer's point-of-view.

\section{Remit}
\label{sec:remit}

The remit will summarize the requirements provided in the document \textit{Systems Programming \& Scripting (2010/2011) Assessment Two} and list all assumptions made in respect to a certain requirement.

For later reference throughout the document, the requirements will be divided into seven groups (\ac{WR}, \ac{HPR}, \ac{FR}, \ac{HR}, \ac{PR}, \ac{GR}, \ac{AR}) and a unique identifier will be assigned to each requirement.

\begin{description}
\item[WR01:] Send \ac{HTTP} request messages for \ac{URL}s typed in by the user.
\item[WR02:] Receive \ac{HTTP} responses for the send requests.
\item[WR03:] Display received \ac{HTML}-code.
\item[WR04:] Display the \ac{HTML}-code, when the received message is either a \texttt{200}, \texttt{400}, \texttt{403} or \texttt{404} status-code.
\item[HR01:] Allow setting and editing of a homepage-\ac{URL}.
\item[HR02:] Load homepage on application start-up.
\item[FR01:] Allow adding and deleting of a \ac{URL} to a list of favourites. Allow editing of favourites present on the list.
\item[FR02:] Allow the specification of a name for a favourite.
\item[FR03:] Request a favourite's \ac{URL} when the favourite is activated.
\item[FR04:] Load all favourites on application start-up.
\item[HR01:] All pages that are requested shall be saved in a history.
\item[HR02:] Request a \ac{URL} when an entry in the history is activated.
\item[HR03:] Load history on application start-up.
\item[PR01:] Allow printing of the currently displayed web page.
\item[GR01:] Provide a \ac{GUI} for the actions specified in the requirements.
\item[GR02:] Use menus, buttons and short cuts to increase \textit{accessibility}\footnote{Assumption: accessibility can be enhanced by using a standard layout common in Windows environments.}.
\item[AR01:] Utilize multi-threading to keep the application responsive.
\item[AR02:] Allow requesting of multiple web pages simultaneously.
\end{description}