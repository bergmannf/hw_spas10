%************************************************
\chapter{Requirement's checklist}\label{ch:requirements} % $\mathbb{ZNR}$
%************************************************

From the requirements stated in \autoref{sec:remit}, the following were fulfilled:

\begin{description}
\item[MSI01:] Implemented in StockItem class with getters and setters.
\item[MSI02:] Implemented in a manager-class that allow adding and deleting.
\item[MSI03:] Implemented as StockItem class.
\item[MSI04:] Implemented in manager-class.
\item[MBA01:] Implemented in BankAccount class with getters and setters.
\item[MBA02:] Fake-method for ordering: will adjust account balance, but not transfer money.
\item[MBA03:] Implemented in manager-class: takes care of Atomicity of request.
\item[MDA01:] Implemented in FileHandler-class and StockItem-class.
\item[MDA02:] Implemented in FileHandler-class.
\item[MDA03:] Implemented in StockItem-class.
\item[MDA04:] Implemented in StockItem-class.
\item[MDA04:] Implemented in StockItem-class.
\item[MDA05:] Verified via testing.
\item[GUI01:] Implemented via WinForms.
\item[GUI02:] Implemented via WinForms.
\end{description}

Apart from fulfilling these requirements the following features were implemented as well to improve the user-experience of the program:

\begin{description}
\item[Error Notification:] Upon entering invalid information the user will be informed about the mistakes.
\item[BankAccount Persistence:] It is possible to import and export bank accounts as well.
\item[Order quantity:] It is possible to order a certain quantity instead of always ordering the required number of items.
\end{description}