%************************************************
\chapter{Developer Guide}\label{ch:developer_guide} % $\mathbb{ZNR}$
%************************************************

To allow further development of the application, certain design decisions from \autoref{ch:design} will be described in greater depth:

\section{UserInterface}
\label{sec:design_user_interface}

The user interface package holds the WinForms representation of a possible \ac{GUI}\footnote{It is a \textit{possible} \ac{GUI}, as  another one should - due to the decoupling of view and logic - be easily realisable by implementing the \texttt{interfaces} of the \texttt{ApplicationLogic} package.}.

The \texttt{MainWindow} holds a reference to the \texttt{presenter} and the \texttt{model}.

The \texttt{presenter} handles all events that need more logic than just changing the view's appearance\footnote{E.g. enabling/disabling buttons, changing the color of fields, showing a new window.}.

The \texttt{model}-reference is used to set-up the data-binding in the application:

\begin{lstlisting}[caption=Data Binding of \texttt{view} and \texttt{model}]
private void SetUpDataBindings()
        {
            stockItemsListBox.DataSource = _Model.StockItems;
            stockItemsListBox.DisplayMember = "Name";

            /*
             * The datasourceupdatemode is set to "Never".
             * This leads to the ability to enforce the use of the presenter to update the values in the model.
             * This way the validation errors can be handled by the presenter thus leading to better seperation of concerns.
             */
            stockCodeTextBox.DataBindings.Add("Text", _Model.StockItems, "StockCode", false, DataSourceUpdateMode.Never);
            itemNameTextBox.DataBindings.Add("Text", _Model.StockItems, "Name", false, DataSourceUpdateMode.Never);
            supplierNameTextBox.DataBindings.Add("Text", _Model.StockItems, "SupplierName", false, DataSourceUpdateMode.Never);
            currStockTextBox.DataBindings.Add("Text", _Model.StockItems, "CurrentStock", false, DataSourceUpdateMode.Never);
            reqStockTextBox.DataBindings.Add("Text", _Model.StockItems, "RequiredStock", false, DataSourceUpdateMode.Never);
            priceTextBox.DataBindings.Add("Text", _Model.StockItems, "UnitCost", false, DataSourceUpdateMode.Never);

            bankAccountsListBox.DataSource = _Model.BankAccounts;
            bankAccountsListBox.DisplayMember = "AccountNumber";

            accountNumberTextBox.DataBindings.Add("Text", _Model.BankAccounts, "AccountNumber", false, DataSourceUpdateMode.Never);
            nameTextBox.DataBindings.Add("Text", _Model.BankAccounts, "Surname", false, DataSourceUpdateMode.Never);
            balanceTextBox.DataBindings.Add("Text", _Model.BankAccounts, "Balance", false, DataSourceUpdateMode.Never);
        }
\end{lstlisting}

Noteworthy is the use of \texttt{DataSourceUpdateMode.Never}. This guarantees that changes from the \ac{GUI} are not propagated to the model via data-binding, but that we can pass them through the presenter and keep the separation between view and model intact.

Another important architectural aspect of the \texttt{view} is the implementation of necessary interfaces for the presenter: instead of passing all attributes with a method call, the presenter will expect the view to implement a certain interface through which it can access the needed attributes:

\lstinputlisting[caption=Example interface \texttt{IStockItemView}]{\logicRoot/Interfaces/IStockItemView.cs}

The \texttt{MainWindow} implements three of these presenter-related interfaces: \texttt{IStockItemView}, \texttt{IBankAccountView}, \texttt{ICongregateView}. The first two guarantee the presenter that it can access all attributes needed to update an item or bank account. The later view provides the following methods and properties:

\lstinputlisting[caption=Interface \texttt{ICongregateView}]{\logicRoot/Interfaces/ICongregateView.cs}

It allows to delete items and bank accounts, as well as provide the necessary application logic to order items, deposit and withdraw money as well as method to display possible validation errors.

Should new views be added an interface should be provided that guarantees the separation between view and presenter, thus allowing the possible reuse of the presenter across multiple \ac{GUI}s.

A problem arising from the .NET architecture is that only the \ac{GUI} project provides a settings file. This leads to the fact that the WinForms-\ac{GUI} has to handle the loading and saving of user-preferences (the file paths to the bank accounts and stock items files).

\section{ApplicationLogic}
\label{sec:application_logic}

The application logic project hosts the \texttt{presenter(s)} as well as the \texttt{models}.

As has been pointed auto in \autoref{ch:design}, the \texttt{AppDataManager}-class works as a \textit{facade} for the rest of the model.

Noteworthy implementations in this package are the \texttt{file handler} and the realisation of \texttt{error handling}.

\subsection{File handler}
\label{subsec:filehandler}

The \texttt{FileHandler}-class utilizes the concept of generics: this way it is possible to reuse this class for multiple classes that need to be persisted.

To allow the serialization and de-serialization-logic to be separated from the file-handling logic, the FileHandler-class requires all classes that need to be persisted to implement the \texttt{ICSVSerializable}-interface:

\lstinputlisting[caption=Interface \texttt{ICSVSerializable}]{\logicRoot/Interfaces/ICSVSerializable.cs}

Due to this fact - both - the \texttt{StockItem} and the \texttt{BankAccount}-class implement this interface.

\subsection{Error handling}
\label{subsec:error_handling}

Error handling is achieved by the separate classes \texttt{ErrorMessageCollection} and \texttt{ErrorMessage}.

On validation error-messages will be added to an error-message-collection that can be accessed from the presenter to display the occurred errors.

As an example the code of the stock item's \texttt{Validate()} method, as well as a sequence-diagram of the calling sequence is shown:

\begin{lstlisting}[caption=Validate method of StockItem]
public static bool Validate(String stockCode, String name, String supplierName, double unitCost, int required, int currentStock)
        {
            if (String.IsNullOrEmpty(stockCode) || !IsValidStockCode(stockCode))
            {
                ErrorMessages.Add(new ErrorMessage("Need a stockcode that adheres to the stockcode format: 4 numbers."));
            }
            if (String.IsNullOrEmpty(name))
            {
                ErrorMessages.Add(new ErrorMessage("Need an item name."));
            }
            if (String.IsNullOrEmpty("supplierName"))
            {
                ErrorMessages.Add(new ErrorMessage("Need a supplier name."));
            }
            if (unitCost < 0.0)
            {
                ErrorMessages.Add(new ErrorMessage("Unit costs must be greater or equal 0."));
            }
            if (required < 0)
            {
                ErrorMessages.Add(new ErrorMessage("Required must be greater or equal 0."));
            }
            if (currentStock < 0)
            {
                ErrorMessages.Add(new ErrorMessage("Current must be greater or equal 0."));
            }
            return ErrorMessages.Count == 0;
        }
\end{lstlisting}

\begin{figure}[H]
\begin{center}
\includegraphics[width=\textwidth]{gfx/error_handling.png}
\caption{Sequence diagram of input validation}
\label{fig:error_handling}
\end{center}
\end{figure}